\subsection{Exercise \#9 - Barbara Bolognese, Lars Kastner, \& Julie Rana}

\subsubsection{Problem}
State the \textit{Hodge Index Theorem}. Verify this theorem for cubic surfaces
in $\P^3$, by explicitly computing the matrix for the intersection pairing.


\subsubsection{Answer}

\paragraph{Statement}

\begin{theorem}
The signature of the intersection pairing on $H^2(X,\R)$ equals $\sum_{p,q}(-1)^ph^{p,q}(X)$.
\end{theorem}

Explanation: If the the pairing is given by a symmetric matrix $A$, then the
signature equals the number of positive eigenvalues minus the number of
negative eigenvalues.

\paragraph{Hodge diamond}
To verify this theorem, we will compute the Hodge diamond of a cubic surface $X$.
\[
\begin{array}{ccccc}
&&h^{00}&&\\
& h^{10} && h^{01} & \\
h^{20} && h^{11} && h^{02}\\
& h^{21} && h^{12} & \\
&&h^{22}&&\\
\end{array}
\]
We have the following symmetries:
\begin{enumerate}
\item $h^{pq}=h^{n-q,n-p}$ because of Serre duality.
\item $h^{pq}=h^{qp}$ because $H^{pq}(X)$ is the complex conjugate $H^{qp}(X)$.
\end{enumerate}
So we just need to compute the upper left triangle of the Hodge diamond.
\begin{itemize}
\item[$h^{00}$:] We have $h^{00}=h^0(X,\O_X)=1$, since $X$ is compact connected.
\item[$h^{01}$:] $h^{01}$ is zero, because the Lefschetz hyperplane theorem implies that a hypersurface in $\P^n$ is simply connected, hence its first homology/cohomology group is also zero. Therefore $0=b_1(X)=2h^{01}$.
\item[$h^{02}$:] Since $\bigwedge^2\Omega_X=\omega_X$, we want to know $h^0(X,\omega_X)$ (use symmetry). Using the adjunction formula we obtain
\[
\omega_X\cong (\omega_{\P^3}\otimes \O(X))|_{X}\cong \O_X(-1),
\]
since $\omega_{\P^3}\cong\O(-4)$ and $X$ has degree $3$. But $\O_X(-1)$ does not have any global sections, so $h^{02}=0$.
\item[$h^{11}$:] We have $\chi_{top}(X)=\sum_{i=0}^4(-1)^{i+1}b_i$. The $b_i$ are sums of the Hodge numbers, namely the rows in the Hodge diamond. Inserting the known numbers, we obtain
\[
\chi_{top}(X)\ =\ h^{11} + 2.
\]
There are two ways to compute the holomorphic Euler characteristic $\chi(\O_X)$, the second way involves $\chi_{top}(X)$.
\begin{enumerate}
\item 
\[
1=\chi(\O_X)=h^0(\O_X)-h^1(\O_X) + h^2(\O_X) = h^{00} - h^{01} + h^{02} = h^2(X,\O_X).
\]
\item 
\begin{align*}
\chi(\O_X) = \int_X\ch(\O_X)\td(\O_X)\\
=\int_X(1,0,0)(1,\frac{c_1(X)}{2},\frac{c_1^2(X) + c_2(X)}{12}) = \int_X\frac{c_1^2(X) + c_2(X)}{12}\\
=\frac{1}{12}(K_X^2+\chi_{top}(X)).
\end{align*}
We have $K_X=-H$ for $H$ a hyperplane section and hence $K_X^2=3$.
\end{enumerate}
This yields $\chi_{top}(X)=9$ and hence $h^{11}=7$.
\end{itemize}
This gives the Hodge diamond
\[
\begin{array}{ccccc}
&&1&&\\
& 0 && 0 & \\
0 && 7 && 0\\
& 0 && 0 & \\
&&1&&\\
\end{array}
\]
and for the signature we get $-5$. Denote by $a$ the number of positive eigenvalues of $A$ and by $b$ the number of negative eigenvalues of $A$. We now have the following two equations
\[
a-b = -5\ \mbox{ and }\ a+b = 7.
\]
Thus $a=1$ and $b=6$.

\paragraph*{Intersection pairing}
Every cubic surface can be given as the blowup of $\P^2$ in six points. Hartshorne Prop 4.8 explicitly gives a basis for $\Pic(X)$. Using this basis, the intersection form is:
\[
\left(
\begin{array}{ccccccc}
1 & 0 & 0 & 0 & 0 & 0 & 0 \\
0 & -1 & 0 & 0 & 0 & 0 & 0 \\
0 & 0 & -1 & 0 & 0 & 0 & 0 \\
0 & 0 & 0 & -1 & 0 & 0 & 0 \\
0 & 0 & 0 & 0 & -1 & 0 & 0 \\
0 & 0 & 0 & 0 & 0 & -1 & 0 \\
0 & 0 & 0 & 0 & 0 & 0 & -1 \\
\end{array}
\right)
\]
