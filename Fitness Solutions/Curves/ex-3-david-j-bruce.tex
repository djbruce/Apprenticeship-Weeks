
\subsection{Exercise \#3 - David J. Bruce}

In order to understand, and eventually find equations for, when two plane conics $C_1$ and $C_2$ are tangent let us set up the following incidence correspondence:
\[
\cC:=\left\{(C_1, C_2, p)\quad  \big|\quad \begin{matrix}\text{$C_1$ and $C_2$ plane conics}\\ \text{$p\in C_1\cap C_2$ and $T_pC_1=T_pC_2$}\end{matrix} \right\}\subset\P^5\times\P^5\times\P^2,
\]
where we think of our coordinates as $([s_0:s_1:s_2:s_3:s_4:s_5],[t_0:t_1:t_2:t_3:t_4:t_5],[x:y:z])$. Notice we are thinking of the space of plane conics as the $\P^5$ given by its coefficients. To find equations for this incidence correspondence $\cC$ notice that we essentially have three conditions:
\begin{enumerate}[(i)]
\item the point $p=[x:y:z]$ lies on $C_1$ meaning that $s_0x^2+s_1y^2+s_2z^2+s_3xy+s_4xz+s_5yz=0$,
\item the point $p=[x:y:z]$ lies on $C_2$ meaning that $t_0x^2+t_1y^2+t_2z^2+t_3xy+t_4xz+t_5yz=0$,
\item the curves $C_1$ and $C_2$ are tangent at the point $p$.
\end{enumerate}
This last condition is the most complicated, but since we know if we let $J_p(C_i)$ be the Jacobian matrix of $C_i$ evaluated at $p$ then 
\begin{align*}
T_pC_1&=\P\left(\ker J_p(C_1)\right)=\P\left(\ker\begin{pmatrix} 2s_0x+s_3y+s_4z & 2s_1y+s_3x+s_5z & 2s_2z+s_4x+s_5y\end{pmatrix}\right)\\
T_pC_2&=\P\left(\ker J_p(C_1)\right)=\P\left(\ker\begin{pmatrix} 2t_0x+t_3y+t_4z & 2t_1y+t_3x+t_5z & 2t_2z+t_4x+t_5y\end{pmatrix}\right)\\
\end{align*}
it reduces to linear algebra. In particular, we have reduced to the question when do $J_p(C_1)$ and $J_p(C_1)$ define the same kernel, which is equivalent to saying the above gradient vectors are linearly dependent. Hence the equations corresponding to condition (iii) are given by:
\[
\text{Minors}_{2}\left(\begin{pmatrix}
2s_0x+s_3y+s_4z & 2s_1y+s_3x+s_5z & 2s_2z+s_4x+s_5y\\
2t_0x+t_3y+t_4z & 2t_1y+t_3x+t_5z & 2t_2z+t_4x+t_5y
\end{pmatrix}\right).
\]
Combining these equations with the equations of type (i) and (ii) it would seem that the incidence correspondence $\cC$ is given by the following ideal:
\[
I=\left\langle \begin{matrix}
s_0x^2+s_1y^2+s_2z^2+s_3xy+s_4xz+s_5yz,\\
2t_0x+t_3y+t_4z + 2t_1y+t_3x+t_5z+2t_2z
\end{matrix}
\right\rangle+\text{Minors}_{2}\left(\begin{pmatrix}
2s_0x+s_3y+s_4z & 2s_1y+s_3x+s_5z & 2s_2z+s_4x+s_5y\\
2t_0x+t_3y+t_4z & 2t_1y+t_3x+t_5z & 2t_2z+t_4x+t_5y
\end{pmatrix}\right),
\]
however, this is not quite true. In particular, we failed to ensure that the point $p$ and our conics $C_1$ and $C_2$ are actually well-defined subsets of projective space. That is to say we never included restrictions to eliminate the points where $x=y=z=0$ or $s_0=s_1=s_2=s_3=s_4=s_5=0$ or $t_0=t_1=t_2=t_3=t_4=t_5=0$ from our correspondence despite these points not corresponding to actual points or conics in $\P^2$. The remedy for this is to saturate our ideal with respects to the ideals $\langle x,y,z\rangle$, $\langle s_0,s_1,s_2,s_3,s_4,s_5\rangle$ and $\langle t_0,t_1,t_2,t_3,t_4,t_5\rangle$. (Note we must saturate by each ideal one at a time.) The resulting ideal, call it $J$, now defines the correspondence $\cC$. 

Now the incidence correspondence $\cC$ carries a natural projection:
\[
\begin{tikzcd}
\pi:\cC\rar{}& \P^5\times \P^5
\end{tikzcd}
\]
given by sending a tuple $(C_1,C_2,p)$ to the pair of conics $(C_1, C_2)$. Moreover, by construction the image of $\pi$ is precisely the loci of conics that are tangent. So in order to find the Tact invariant we are left to compute defining equations for the image $\pi(\cC)\subset\P^5\times\P^5$.

Given the ideal $J$ this can an easily be accomplished via Macaulay2 via the \verb+eliminate+ command. In particular, using the command:
\[
A=\verb+eliminate({x,y,z}, J)+
\]
will produce the ideal $A$ generated by all the elements of $J$ not using the variables $x,y,z$. This resulting ideal clearly defines $\pi(\cC)$, and so should hopefully be generated by the Tact invariant. Implementing this construction of $J$ and $A$ in Macaulay2 we find that $A$ is generated by a bi-degree $(6,6)$ polynomial with 3210 terms, the first few being:
\[
s_0^2s_3^2s_4^2t_t^6-4s_0^3s_1s_4^2t_5^6-4s_0^3s_2s_3^2t_5^6-2s_0^2s_3s_4^2s_5t_3t_5^5+8s_0^3s_1s_3s_4t_2t_5^2+\cdots
\]

\subsubsection{Complete Code - (Macaulay2)}

\begin{verbatim}

restart

S = QQ[x,y,z,s0,s1,s2,s3,s4,s5,t0,t1,t2,t3,t4,t5]

f = s0*x^2+s1*y^2+s2*z^2+s3*x*y+s4*x*z+s5*y*z;
g = t0*x^2+t1*y^2+t2*z^2+t3*x*y+t4*x*z+t5*y*z;

M = matrix{
	{2*s0*x+s3*y+s4*z, 2*s1*y+s3*x+s5*z, 2*s2*z+s4*x+s5*y},
	 {2*t0*x+t3*y+t4*z, 2*t1*y+t3*x+t5*z, 2*t2*z+t4*x+t5*y}};
	 
J = ideal(f,g)+minors(2,M);
I = saturate(J, ideal(x,y,z));

A = eliminate({x,y,z},I);

\end{verbatim}
