\documentclass[11pt]{amsart}
\usepackage{amstext,amsfonts,amssymb,amscd,amsbsy,amsmath,verbatim,fullpage,mathrsfs}
\usepackage[letterpaper, margin=.7in]{geometry}

\usepackage[alphabetic,lite,backrefs]{amsrefs} % for bibliography
%\usepackage[bookmarks, colorlinks=true, linkcolor=blue, citecolor=blue, urlcolor=blue, pagebackref=true]{hyperref}
\usepackage{ifthen}
\usepackage{amsthm}
\usepackage{latexsym}
\usepackage[all]{xy}
\usepackage{enumerate}
\usepackage{pdflscape}
\usepackage[usenames,dvipsnames,svgnames,table]{xcolor}
\usepackage{bbm}									     					% Creates nice blackboard bold 1
\usepackage{tikz}									    					% Allows us to create diagrams
\usetikzlibrary{positioning}
\tikzset{main node/.style={circle,fill=blue!20,draw,minimum size=.65cm,inner sep=0pt},
            }
\tikzset{gre node/.style={plain},
            }            

\usetikzlibrary{matrix}													% Calls useful TikZ libraries
\usetikzlibrary{shapes}													% Calls useful TikZ libraries
\usepackage{tikz-cd}									    					% Nice for creating commutative diagrams
\usepackage{mathtools}

\usepackage{float}	
\usepackage{parskip}

\usepackage{titlesec}								     					% Allows one to use nice section titles
\titleformat{\section}[block]{\large\scshape\bfseries\filcenter}{\thesection.}{1em}{}		% Creates section titles
\titleformat{\subsection}[hang]{\large\scshape\bfseries}{\thesubsection}{1em}{}			% Creates subsection titles


\usepackage[titles]{tocloft}								     					% Creates table of fancy contents
\renewcommand{\contentsname}{}	     					% Renames and centers title of ToC

\usepackage[bookmarks,linktoc=all,pagebackref=true]{hyperref}						    					% Creates the linkable ToC
\hypersetup{
    colorlinks,
    citecolor=blue,
    filecolor=black,
    linkcolor=blue,
    urlcolor=black
}

\usepackage{kpfonts}								    					% This font allows you to use both bold and scshape


\newtheorem{lemma}{Lemma}[section]
\newtheorem{theorem}[lemma]{Theorem}
\newtheorem{propo}[lemma]{Proposition}
\newtheorem{prop}[lemma]{Proposition}
\newtheorem{cor}[lemma]{Corollary}
\newtheorem{conj}[lemma]{Conjecture}
\newtheorem{claim}[lemma]{Claim}
\newtheorem{claim*}{Claim}
\newtheorem{thm}[lemma]{Theorem}
\newtheorem*{thmA}{Theorem A}
\newtheorem*{thmB}{Theorem B}
\newtheorem*{thmC}{Theorem C}
\newtheorem{defn}[lemma]{Definition} 
\newtheorem{notation}[lemma]{Notation} 
\newtheorem{ex}[lemma]{Exercise}
\newtheorem{question}[lemma]{Question}
\newtheorem{assumption}[lemma]{Assumption}
\newtheorem{principle}[lemma]{Principle}
\newtheorem{heuristic}[lemma]{Heuristic}

\theoremstyle{remark}
\newtheorem{remark}[lemma]{Remark}
\newtheorem{example}[lemma]{Example}


% Commands
\newcommand{\initial}{\operatorname{in}}
\newcommand{\NF}{\operatorname{NF}}
\newcommand{\Hilb}{\operatorname{Hilb}}
\newcommand{\depth}{\operatorname{depth}}
\newcommand{\reg}{\operatorname{reg}}
\newcommand{\cone}{\textbf{cone}}

\newcommand{\length}{\operatorname{length}}
\newcommand{\coker}{\operatorname{coker}}
\newcommand{\adeg}{\operatorname{adeg}}
\newcommand{\pdim}{\operatorname{pdim}}
\newcommand{\Spec}{\operatorname{Spec}}
\newcommand{\Ext}{\operatorname{Ext}}
\newcommand{\Tor}{\operatorname{Tor}}
\newcommand{\LT}{\operatorname{LT}}
\newcommand{\im}{\operatorname{im}}
\newcommand{\NS}{\operatorname{NS}}
\newcommand{\Frac}{\operatorname{Frac}}
\newcommand{\Char}{\operatorname{char}}
\newcommand{\Proj}{\operatorname{Proj}}
\newcommand{\id}{\operatorname{id}}
\newcommand{\Div}{\operatorname{Div}}
\newcommand{\tr}{\operatorname{tr}}
\newcommand{\Res}{\operatorname{Res}}
\newcommand{\Tr}{\operatorname{Tr}}
\newcommand{\Supp}{\operatorname{Supp}}
\newcommand{\ann}{\operatorname{ann}}
\newcommand{\Gal}{\operatorname{Gal}}
\newcommand{\Aut}{\operatorname{Aut}}
\newcommand{\Span}{\operatorname{span}}


\newcommand{\Pic}{\operatorname{Pic}}
\newcommand{\QQbar}{{\overline{\mathbb Q}}}
\newcommand{\Br}{\operatorname{Br}}
\newcommand{\Bl}{\operatorname{Bl}}
\newcommand{\Cox}{\operatorname{Cox}}
\newcommand{\getsr}{\operatorname{Tor}}
\newcommand{\diam}{\operatorname{diam}}
\newcommand{\Hom}{\operatorname{Hom}} %done
\newcommand{\sheafHom}{\mathcal{H}om}
\newcommand{\Gr}{\operatorname{Gr}}
\newcommand{\rank}{\operatorname{rank}}
\newcommand{\codim}{\operatorname{codim}}
\newcommand{\Sym}{\operatorname{Sym}} %done
\newcommand{\GL}{{GL}}
\newcommand{\Prob}{\operatorname{Prob}}
\newcommand{\Density}{\operatorname{Density}}
\newcommand{\Tot}{\operatorname{Tot}}

\newcommand{\Syz}{\operatorname{Syz}}
\newcommand{\defi}[1]{\textsf{#1}} % for defined terms

\newcommand{\arrow}{\xrightarrow}							% tuple
\newcommand{\xn}{x_1,\ldots,x_n}							% tuple
\newcommand{\xN}{x_0,\ldots,x_n}							% tuple
\newcommand{\zn}{z_1,\ldots,z_n}							% tuple
\newcommand{\vn}{v_1,\ldots,v_n}							% tuple


%%%%%%%%%%%%%%%%%%%%%%%%%%%%%% Letters  %%%%%%%%%%%%%%%%%%%%%%%%%%%%%%%%%%%%%%%%%%%%
%%%%%%%%%%%%%%%%%%%%%%%%%%%%%%%%%%%%%%%%%%%%%%%%%%%%%%%%%%%%%%%%%%%%%%%%%%%%%%
\newcommand{\fF}{\mathbf F}
\newcommand{\kK}{\mathbf K}
\newcommand{\BS}{\mathbf S}
\newcommand{\qQ}{\mathbf Q}
\newcommand{\rR}{\mathbf R}

\newcommand{\ff}{\mathbf f}
\newcommand{\kk}{\mathbf k}
\renewcommand{\O}{\mathcal{O}}

%%%% Caligraphic Fonts - i.e. ????. %%%%%
\newcommand{\cA}{\mathcal{A}}
\newcommand{\cB}{\mathcal{B}}
\newcommand{\cC}{\mathcal{C}}
\newcommand{\cD}{\mathcal{D}}
\newcommand{\cE}{\mathcal{E}}
\newcommand{\cF}{\mathcal{F}}
\newcommand{\cG}{\mathcal{G}}
\newcommand{\cH}{\mathcal{H}} 
\newcommand{\cI}{\mathcal{I}}
\newcommand{\cJ}{\mathcal{J}}
\newcommand{\cK}{\mathcal{K}}
\newcommand{\cL}{\mathcal{L}}
\newcommand{\cM}{\mathcal{M}}
\newcommand{\cN}{\mathcal{N}}
\newcommand{\cO}{\mathcal{O}}
\newcommand{\cP}{\mathcal{P}}
\newcommand{\cQ}{\mathcal{Q}}
\newcommand{\cR}{\mathcal{R}}
\newcommand{\cS}{\mathcal{S}}
\newcommand{\cT}{\mathcal{T}}
\newcommand{\U}{\mathcal{U}} 		% Notice this is different
\newcommand{\cV}{\mathcal{V}}
\newcommand{\cW}{\mathcal{W}}
\newcommand{\cX}{\mathcal{X}}
\newcommand{\cY}{\mathcal{Y}}
\newcommand{\cZ}{\mathcal{Z}}

%%%% Blackboard Fonts - i.e. Real Numbers, Integers, etc. %%%%%
\newcommand{\A}{\mathbb{A}}
\newcommand{\B}{\mathbb{B}}
\newcommand{\C}{\mathbb{C}}
\renewcommand{\D}{\mathbb{D}}
\newcommand{\E}{\mathbb{E}}
\newcommand{\F}{\mathbb{F}}
\newcommand{\G}{\mathbb{G}}
\renewcommand{\H}{\mathbb{H}} 
\newcommand{\I}{\mathbb{I}}
\newcommand{\J}{\mathbb{J}}
\newcommand{\K}{\mathbb{K}}
\renewcommand{\L}{\mathbb{L}}
\newcommand{\M}{\mathbb{M}}
\newcommand{\N}{\mathbb{N}}
\newcommand{\bO}{\mathbb{O}}		% Notice this is \bO
\renewcommand{\P}{\mathbb{P}}
\newcommand{\Q}{\mathbb{Q}}
\newcommand{\R}{\mathbb{R}}
\renewcommand{\S}{\mathbb{S}}
\newcommand{\T}{\mathbb{T}}
\newcommand{\bU}{\mathbb{U}}		% Notice this is \bU
\newcommand{\V}{\mathbb{V}}
\newcommand{\W}{\mathbb{W}}
\newcommand{\X}{\mathbb{X}}
\newcommand{\Y}{\mathbb{Y}}
\newcommand{\Z}{\mathbb{Z}}

 %%%% Sarif Fonts - i.e. ???? %%%%%
\newcommand{\sA}{\mathsf{A}}
\newcommand{\sB}{\mathsf{B}}
\newcommand{\sC}{\mathsf{C}}
\newcommand{\sD}{\mathsf{D}}
\newcommand{\sE}{\mathsf{E}}
\newcommand{\sF}{\mathsf{F}}
\newcommand{\sG}{\mathsf{G}}
\newcommand{\sH}{\mathsf{H}} 
\newcommand{\sI}{\mathsf{I}}
\newcommand{\sJ}{\mathsf{J}}
\newcommand{\sK}{\mathsf{K}}
\newcommand{\sL}{\mathsf{L}}
\newcommand{\sM}{\mathsf{M}}
\newcommand{\sN}{\mathsf{N}}
\newcommand{\sO}{\mathsf{O}}
\newcommand{\sP}{\mathsf{P}}
\newcommand{\sQ}{\mathsf{Q}}
\newcommand{\sR}{\mathsf{R}}
\newcommand{\sS}{\mathsf{S}}
\newcommand{\sT}{\mathsf{T}}
\newcommand{\sU}{\mathsf{U}} 
\newcommand{\sV}{\mathsf{V}}
\newcommand{\sW}{\mathsf{W}}
\newcommand{\sX}{\mathsf{X}}
\newcommand{\sY}{\mathsf{Y}}
\newcommand{\sZ}{\mathsf{Z}}
 
 %%%% Fraktur Fonts - i.e. maximal ideals, prime ideals, etc. %%%%%
\newcommand{\cl}{\mathfrak{cl}}
\newcommand{\g}{\mathfrak{g}}
\newcommand{\h}{\mathfrak{h}}
\newcommand{\m}{\mathfrak{m}}
\newcommand{\n}{\mathfrak{n}}
\newcommand{\p}{\mathfrak{p}}
\newcommand{\q}{\mathfrak{q}}
\renewcommand{\r}{\mathfrak{r}}
\newcommand{\doot}{\bullet}


\newcommand{\bl}[1]{{\color{blue}#1}}
\newcommand{\re}[1]{{\color{red}#1}}
\newcommand{\br}[1]{{\color{RawSienna}#1}}
\newcommand{\gr}[1]{{\color{ForestGreen}#1}}

\newcommand{\gre}[1]{{\color{ForestGreen}\tiny#1}}
\newcommand{\red}[1]{{\color{red}\tiny#1}}

\newcommand{\stk}[2]{{\genfrac{}{}{0pt}{0}{\re{#1}}{\genfrac{}{}{0pt}{0}{\oplus}{\bl{#2}}}}}

\newcommand{\ve}[1]{\ensuremath{\mathbf{#1}}}
\newcommand{\chr}{\ensuremath{\operatorname{char}}}
\newcommand\commentr[1]{{\color{red} \sf [#1]}}
\newcommand\commentb[1]{{\color{blue} \sf [#1]}}
\newcommand\commentm[1]{{\color{magenta} \sf [#1]}}
\newcommand{\daniel}[1]{{\color{blue} \sf $\clubsuit\clubsuit\clubsuit$ Daniel: [#1]}}
\newcommand{\deej}[1]{{\color{red} \sf $\spadesuit\spadesuit\spadesuit$ DJ: [#1]}}

\setcounter{MaxMatrixCols}{20}

\title{Apprenticeship Week - Fitness Exercise}

\author{Madeline Brandt}
\address{Department of Mathematics, University of California, Berkeley, CA}
\email{\href{mailto:djbruce@math.wisc.edu}{djbruce@math.wisc.edu}}
\urladdr{\url{http://math.wisc.edu/~djbruce/}}

\author{David J. Bruce}
\address{Department of Mathematics, University of Wisconsin, Madison, WI}
\email{\href{mailto:brandtm@berkeley.edu}{brandtm@berkeley.edu}}
\urladdr{\url{https://math.berkeley.edu/~brandtm/}}

\author{Alessandro Oneto}
\address{Matematiska Institutionen, Stockholms Universitet}
\email{\href{mailto:oneto@math.su.se}{oneto@math.su.se}}
\urladdr{\url{https://staff.math.su.se/oneto/}}



\begin{document} 
\maketitle
\tableofcontents

%%%%%%%%%%%%%%%%%%%%%%%%%%%%%%%%%%%%%%%%%%%%%%%%%%%%%%%%%%%%%%%%%%%%%%%%%%%%%%

\section{Wednesday, August 24, 2016}

In order to understand the incidence correspondence of the 27 lines on a smooth cubic surface in $\P^3$ over an algebraically closed field we will utilize the following result, which is stated in Remark 4.7.1 of \cite{hartshorne}.

\begin{theorem}
Any smooth cubic surface $X$ in $\P^3$ over an algebraically closed field is isomorphic to the blow-up of $\P^2$ at six points $p_1,\ldots,p_6$ where no three of the $p_i$ are colinear and all six are not on a cubic.
\end{theorem}

\deej{ADD PROOF}

Using this fact we see two things:
\begin{enumerate}[(i)]
\item the incidence graph of lines on a smooth cubic surface in $\P^3$ (over an algebraically closed field) is well-defined i.e. does not depend on our choice of $X$;
\item this graph can be constructed by considering the lines on $X$ the blow-up of $\P^2$ at six sufficently generic chosen points $p_1,\ldots,p_6$. 
\end{enumerate}
Restricting our attention to the cubic surface $X$ described above we can visualize the lines as shown below:

In particular, from this picture above that our 27 lines come in three flavors:
\begin{enumerate}[(i)]
\item one line, the exceptional divisor $E_i$, over the point $p_i$ 
\item one line $\ell_{i,j}$ coming from the strict transform of the line between $p_i$ and $p_j$
\item one line $C_i$ coming from strict transform of the unique plane conic going through $p_1,\ldots,\hat{p}_i,\ldots,p_6$.
\end{enumerate}
(Note we will often abuse notation and use the same symbol for a curve and its strict transform.) Shift towards trying to compute the incidence graph $G$ of these lines the key fact is the following exercise:

\begin{ex}
Let $C_1$ and $C_2$ be curves in $\P^2$ and $X$ be the cubic surface obtained from blowing $\P^2$ at six general points $\{p_1,\ldots,p_6\}$. Show the strict transforms of $C_1$ and $C_2$ intersect on $X$ if and only if $C_1$ and $C_2$ have a point of intersection outside of $\{p_1,\ldots,p_6\}$.
\end{ex}

In particular, this exercise says that in order to understand intersections on $X$ it is enough to think about intersections on $\P^2$. For example, in the plane the two lines $\ell_{i,j}$ and $\ell_{s,t}$ intersect in exactly one point. Moreover, the intersection $\{i,j\}\cap \{s,t\}$ is non-empty if and only if this point of intersection is one of the $p_i$. Therefore, we see that:
\[
\ell_{i,j}\cap \ell_{s,t}=\begin{cases} \o &\mbox{if } \{i,j\}\cap\{s,t\}\neq\o \\ 
\{*\} & \mbox{else }  \end{cases} 
\]
Similarly, we know that the line $\ell_{i,j}$ and the conic $C_{k}$ intersect in precisely two points. (We shall assume the points $p_i$ have been chosen sufficiently generally so that none of the lines of interest are tangent to any of the conics.) \deej{An example here might be cool.}

Finally, by a similar line of reasoning one can see that:
\[
E_i\cap E_j=\o, \quad\quad E_i\cap \ell_{j,k}=\begin{cases} \{*\} &\mbox{if } i\in \{j,k\} \\ 
\o & \mbox{else }  \end{cases} \quad \quad, \text{and} \quad \quad E_i\cap C_j=\begin{cases} \{*\} &\mbox{if } i\neq j \\ 
\o & \mbox{else }  \end{cases}.
\]

Coding these relations into Macaulay2 we can construct the adjacency matrix of our graph $G$. From this it is easy to verify -- for example by summing the rows of the adjacency matrix -- that the graph is $10$-regular i.e. every vertex has degree 10. Additionally, we can try to visualize it as seen below:
\deej{NEDEDED}

Turning our attention to finding independent sets Macaulay2 shows the maximal ones are of size six. For example, the the six exceptional divisors $E_1,\ldots,E_6$ do not intersect, and so the corresponding vertices are an independent set. In fact, all the maximal independent sets can be classified as follows:

\begin{tabular}{c | c | c}
\textbf{Type} & \textbf{Example} & \textbf{Number} \\ \hline 
$\{E_1,\ldots,E_6\}$ & $\{E_1,\ldots,E_6\}$ & 1 \\ \hline
$\{C_1,\ldots,C_6\}$ &$\{C_1,\ldots,C_6\}$  & 1 \\ \hline 
 $\left\{\{E_i,E_j,E_k,\ell_{a,b},\ell_{c,d},\ell_{e,f}\} \; \bigg| \; i,j,k\not\in \{a,b,c,d,e,f\}\right\}$ & $\{E_1, E_2, E_3, \ell_{4,5}, \ell_{4,6}, \ell_{5,6}\}$ & 20 \\  \hline
$\left\{\{C_i,C_j,C_k,\ell_{a,b},\ell_{c,d},\ell_{e,f}\} \; \bigg| \; i,j,k\not\in \{a,b,c,d,e,f\}\right\}$ & $\{C_1, C_2, C_3, \ell_{4,5}, \ell_{4,6}, \ell_{5,6}\}$ & 20 \\ \hline
 $\left\{\{E_i, C_i, \ell_{a,b},\ell_{c,d},\ell_{e,f},\ell_{g,h}\} \; \bigg| \; \right\}$ & $\{E_1, C_1, \ell_{2,3}, \ell_{2,4}, \ell_{2,5}, \ell_{2,6}\}$ & 30 \\\hline
 & \textbf{Total} & \textbf{72}
\end{tabular}

Similarly, we can also classify the independent sets of size five, i.e. one less than maximal, as follows:

\deej{NEDEDEDED}

Thus, we see that the complex of independent sets of $G$, which we denote $\cC_{\text{Ind}}(G)$, has exactly 72 faces of dimension 5 and 648 face of dimension 4. \deej{Can we say something more from this....} In order to get our ``hands'' on the full complex $\cC_{\text{Ind}}(G)$ we use the fact that the independence complex is the clique complex of complement graph. (Recall the complement of a graph is the graph with the same vertices where two vertices are adjacent if they are not adjacent in the initial graph.) Hence the starting with the adjaceny matrix $A$ for $G$ the following Macaulay2 code, using the Graphs \deej{NEDEDED} package, computes $\cC_{\text{Ind}}(G)$. 
\begin{verbatim}
loadPackage "Graphs"
G = graph A
G' = complementGraph G
S = cliqueComplex G'
\end{verbatim}
Doing this we can confirm that $\cC_{\text{Ind}}(G)$ is five-dimensional, and moreover, its $f$-vector is:
 \[
 \text{fVec}(\cC_{\text{Ind}}(G))=(27, 216,720,1080,648,72).
 \]
 
%%%%%%%%%%%%%%%%%%%%%%%%%%%%%%%%%%%%%%%%%%%%%%%%%%%%%%%%%%%%%%%%%%%%%%%%%%%%%%

\begin{bibdiv}
\begin{biblist}

\bib{eisenbud}{book}{
   author={Eisenbud, David},
   title={Commutative algebra},
   series={Graduate Texts in Mathematics},
   volume={150},
   note={With a view toward algebraic geometry},
   publisher={Springer-Verlag, New York},
   date={1995},
   pages={xvi+785},
%   isbn={0-387-94268-8},
%   isbn={0-387-94269-6},
%   review={\MR{1322960 (97a:13001)}},
%   doi={10.1007/978-1-4612-5350-1},
}
        
\bib{hartshorne}{book}{
   author={Hartshorne, Robin},
   title={Algebraic Geometry},
   series={Graduate Texts in Mathematics},
   volume={52},
   note={With a view toward algebraic geometry},
   publisher={Springer-Verlag, New York},
   date={1977},
   pages={xvi+496},
%   isbn={0-387-94268-8},
%   isbn={0-387-94269-6},
%   review={\MR{1322960 (97a:13001)}},
%   doi={10.1007/978-1-4612-5350-1},
}

\bib{M2}{misc}{
    label={M2},
    author={Grayson, Daniel~R.},
    author={Stillman, Michael~E.},
    title = {Macaulay 2, a software system for research
	    in algebraic geometry},
    note = {Available at \url{http://www.math.uiuc.edu/Macaulay2/}},
}

\bib{mircea}{article}{
   author={Musta\c{t}\u{a}, Mircea},
   title={Zeta functions in algebraic geometry},
   date={2011},
   note = {Available at \url{http://www.math.lsa.umich.edu/~mmustata/zeta_book.pdf}},
%   issn={0002-9327},
%   review={\MR{0065218 (16,398d)}},
}

%http://www.math.lsa.umich.edu/~hochster/615W10/supNoeth.pdf
\end{biblist}
\end{bibdiv}
\end{document}